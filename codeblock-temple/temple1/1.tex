%\documentclass[platex,dvipdfmx]{jlreq}		% jlreqドキュメントクラスをコメントアウト
\documentclass[dvipdfmx]{article}				% articleドキュメントクラスを使用
\usepackage{graphicx}
\usepackage{bxtexlogo}
\usepackage{listings}
\usepackage{here}
\usepackage{url}
\usepackage{listings}
\usepackage{xcolor}

% カラースキームの設定
\definecolor{codegray}{gray}{0.95}
\definecolor{codegreen}{rgb}{0,0.6,0}
\definecolor{codepurple}{rgb}{0.58,0,0.82}
\definecolor{backcolour}{rgb}{0.95,0.95,0.92}

% listingsパッケージのオプション設定
\lstset{
  backgroundcolor=\color{backcolour},
  commentstyle=\color{codegreen},
  keywordstyle=\color{magenta},
  numberstyle=\tiny\color{codegray},
  stringstyle=\color{codepurple},
  basicstyle=\footnotesize\ttfamily,
  breakatwhitespace=false,
  breaklines=true,
  captionpos=b,
  keepspaces=true,
  %numbers=left,
  numbersep=5pt,
  showspaces=false,
  showstringspaces=false,
  showtabs=false,
  tabsize=2
}
\begin{document}





\subsection*{問8}

ガウス分布(下記参照)のプロットを行います。\\
\[
y = \frac{1}{\sqrt{2\pi}} e^{-\frac{x^2}{2}}
\]

\begin{lstlisting}[language=Matlab]
% ガウス分布のグラフをプロット
% y = \frac{1}{\sqrt{2\pi}} \cdot e^{-\frac{x^2}{2}}

prob8 = @(x) ((1/(2*pi)) .* exp(-x.^2/2));
range = [-5,5];
fplot(prob8,range)
\end{lstlisting}



\subsection*{問9}
下記に示された積分を行います。
\[
\int_{-\infty}^{\infty} \frac{1}{\sqrt{2\pi}} e^{-\frac{x^2}{2}} \, dx
\]

\begin{lstlisting}[language=Matlab]
prob9 = @(x) (1/(sqrt(2*pi))) .* exp(-x.^2/2);
integral(prob9,-inf,inf)
\end{lstlisting}

















\end{document}
